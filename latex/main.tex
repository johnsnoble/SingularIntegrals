\documentclass{article}
\usepackage{amsmath}
\usepackage{amsfonts}
\usepackage{graphicx} % Required for inserting images
\usepackage{mathtools}

\mathtoolsset{showonlyrefs}

\title{FYP}
\author{Johns Noble}
\date{January 2025}

\begin{document}

\maketitle

\newpage
\tableofcontents
\newpage

\section{Introduction}
\subsection{Cauchy Transform}
\begin{equation} \label{cauchy transform}
C_\Gamma f(z):=\frac{1}{2\pi i}\int_\Gamma \frac{f(t)}{t-z}dt
\end{equation}
This is analytic for $z \not\in \Gamma$. Define Hilbert Transform to be the limits from the right and the left.
\subsection{Orthogonal Polynomials}
\begin{center}
\begin{tabular}{ |c|c|c|c| } 
 \hline
	Family & Notation & Interval & $w(x)$ \\ 
 \hline
	Legendre & $P_n(x)$ & [-1,1] & $1$ \\ 
	Chebyshev (1st) & $T_n(x)$ & [-1,1] & $(1-x^2)^{-1/2}$ \\ 
	Chebyshev (2nd) & $U_n(x)$ & [-1,1] & $(1-x^2)^{1/2}$ \\
	Ultraspherical & $C_n^{(\lambda)}(x),\:\lambda>-\frac{1}{2}$ & [-1,1] & $(1-x^2)^{\lambda-1/2}$ \\
	Jacobi & $P_n^{(\alpha,\beta)}(x),\:\alpha,\beta>-1$ & [-1,1] & $(1-x)^\alpha(1-x)^\beta$ \\
 \hline
\end{tabular}
\end{center}
\section{Log and Stieltjes Transform}
In this section we will consider approaches to compute these weakly singular integrals
$$ \int_Alog||z-t||f(t)dt \qquad \int_A\mathbf{\nabla}log||z-t||f(t)dt $$
\begin{equation}\label{stieltjes transform}
	\mathcal{S}_Af(z):= \int_A\frac{f(t)}{z-t}dt \\
\end{equation}
\begin{equation}\label{log transform}
	\mathcal{L}_Af(z):= \int_Alog(z-t)f(t)dt
\end{equation}
Depending on the type of area which $A$ is we can begin by approximating $f$ using orthogonal polynomials.
\subsection{Transforms across Intervals}
We will try to formulate recurrence relations for these transforms across interval [-1, 1].
We are looking for looking for $\mathcal{S}_{[-1,1]}f(z)$.
Decomposing $f(z) \approx \Sigma_k f_kP_k(z)$ and writing $S_k(z):=\mathcal{S}_{[-1,1]}P_k(z)$ lets us write:
$$\mathcal{S}_{[-1,1]}f(z) \approx \Sigma_k f_kS_k(z)$$
This motivates finding fast methods to compute $S_k(z)$. Log kernels are approached similarly letting $L_k(z):=\mathcal{L}_{[-1,1]}P_k(z)$ and looking for recurrence relations.
\subsubsection*{Stieltjes}
Recall recurrence relation of Legendre Polynomials:
\begin{equation}\label{legendre recurrence}
	xP_k(x) = \frac{k}{2k+1}P_{k-1}(x) + \frac{k+1}{2k+1}P_{k+1}(x)
\end{equation}
Formulate three-term recurrence for their Stieltjes transforms.
\begin{equation}
\begin{split}
	zS_k(z) &= \int_{-1}^{1}\frac{zP_k(t)}{z-t}dt \\
	&= \int_{-1}^{1}\frac{z-t}{z-t}P_k(t)dt+\int_{-1}^{1}\frac{tP_k(t)}{z-t}dt \\
	&= \int_{-1}^{1}P_k(t)dt+\frac{k}{2k+1}\int_{-1}^{1}\frac{P_{k-1}(t)}{z-t}dt+\frac{k+1}{2k+1}\int_{-1}^{1}\frac{P_{k+1}(t)}{z-t}dt \\
	&= 2\delta_{k0}+\frac{k}{2k+1}S_{k-1}(z)+\frac{k+1}{2k+1}S_{k+1}(z) \\
	S_0(z) &= \int_{-1}^{1}\frac{dt}{z-t} = log(z+1)-log(z-1)
\end{split}
\end{equation}
We can extend this to work over a square using the recurrence over intervals:
\begin{equation}
\begin{split}
zS_{k,j}(z) &= z\int_{-1}^1\int_{-1}^1\frac{P_k(s)P_j(t)}{z-(s+it)}dsdt \\
&= \int_{-1}^1zP_j(t)\int_{-1}^1\frac{P_k(s)}{z-it-s}dsdt \\
&= \int_{-1}^1(z-it)P_j(t)S_k(z-it)+itP_j(t)S_k(z-it)dsdt \\
&= \int_{-1}^1P_j(t)(\frac{k}{2k+1}S_{k-1}(z-it)+\frac{k+1}{2k+1}S_{k+1}(z-it)+2\delta_{k0}) \\
&+i(\frac{j}{2j+1}P_{j-1}(t)+\frac{j+1}{2j+1}P_{j+1}(t))S_k(z-it)dsdt \\
&= \frac{k}{2k+1}S_{k-1,j}(z)+\frac{k+1}{2k+1}S_{k+1,j} \\
&+i\frac{j}{2j+1}S_{k,j-1}(z)+i\frac{j+1}{2j+1}S_{k,j+1}+4\delta_{j0}\delta{k0}
\end{split}
\end{equation}
\subsubsection*{Log}
We can begin by connecting log kernel to the Stieltjes kernel. To do this we define:$$S_k^{(\lambda)}(z):=\int_{-1}^{1}\frac{C_k^{(\lambda)}(t)}{z-t}dt$$
We let $F(x) = \int_{-1}^1f(s)ds$ and apply integration by parts on log transform:
\begin{equation}
\begin{split}
	\int_{-1}^1f(t)log(z-t)dt &= [-F(t)log(z-t)]_{-1}^1-\int_{-1}^1\frac{F(t)}{z-t}dt \\
	&= log(z+1)\int_{-1}^1f(t)dt-\int_{-1}^1\frac{F(t)}{z-t}dt
\end{split}
\end{equation}

\section{Polynomial Transforms}
We can begin to consider taking these transforms across different geometries.
Currently we have a way to find these transforms across [-1,1] but we will be trying to use this to solve other geometries.
The first type of geometry we should consider is one where we apply a degree $d$ polynomial transform to the interval:
$$p:[-1,1]\rightarrow \Gamma$$
We will show why the solution to a cauchy transform across this interval is as follows:
\begin{equation}
C_\Gamma f(z) = \Sigma_{j=0}^dC_{[-1,1]}[f\circ p](p_j^{-1}(z))
\end{equation}
Where $p_j^{-1}(z)$ are the $d$ pre-images of $p$.
In order to solve this we will use plemelj.
There are 3 properties that need to hold for a function $\psi: \Gamma \rightarrow \mathbb{C}$ to be a cauchy transform:
\begin{equation}\label{cauchy_conditions}\begin{gathered}
\underset{z\to\infty}{lim}= 0 \\
\psi^+(z)-\psi^-(z)= f(z) \\
\psi\;analytic\;on\;\Gamma 
\end{gathered}\end{equation}

Checking \eqref{cauchy_conditions}.1 we get that $\underset{z\to\infty}{p_j^{-1}(z)} = \infty \implies$
\begin{equation}\begin{split}
\underset{z\to\infty}{lim}C_\Gamma f(z) &= \Sigma_{j=1}^d \underset{z\to\infty}{lim}C_{[-1,1]}(f\circ p)(p_j^{-1}(z)) \\
&= \Sigma_{j=1}^d C_{[-1,1]}(f\circ p)(\underset{z\to\infty}{lim} p_j^{-1}(z)) \\
&= \Sigma_{j=1}^d 0 = 0
\end{split}\end{equation}

Checking \eqref{cauchy_conditions}.2 we need an expression for $\psi^+$ and $\psi^-$.
Let us begin by saying that we are looking for cauchy transform of point $s$ which happens to lie on $\Gamma$.
This means that there is a unique root of $t_k := p_k^{-1}(s) \in [-1,1]$.
TODO: Show that $\underset{z\to s^+}{lim} p_k^{-1}(s) =\underset{z\to p^{-1}(s)^+}{lim}$.
Taking limits of $\psi^+, \psi^-$ gives us:
\begin{equation}\begin{split}
\psi^+(s)&=\underset{z\to s}{lim}\:C_{[-1,1]}(f\circ p)(p_k^{-1}(z)) \\
&+\Sigma_{j\neq k}C_{[-1,1]}(f\circ p)(p_j^{-1}(s)) \\
&=C^+_{[-1,1]}(f\circ p)(p_k^{-1}(z)) \\
&+\Sigma_{j\neq k}C_{[-1,1]}(f\circ p)(p_j^{-1}(s)) \\
\end{split}\end{equation}
We can do a similar thing with $\psi^-$ and putting everything together:
\begin{equation}\begin{split}
\psi^+(s)-\psi^-(s)&=C_{[-1,1]}^+(f\circ p)(p_k^{-1}(s))-C_{[-1,1]}^-(f\circ p)(p_k^{-1}(s)) \\
&= (f\circ p)(p_k^{-1}(s)) = f(s)
\end{split}\end{equation}
In the case where $z \notin \psi, \psi^+=\psi^-$ which is expected since the area in between is analytic

TODO show that condition \eqref{cauchy_conditions}.3 holds
\end{document}

